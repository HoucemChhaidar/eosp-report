\section{Sprint 4 – Stats \& Deployment}

In this section, we first present the Sprint 4 backlog, then detail the design and implementation of key features through screenshots and diagrams.

\subsection{Sprint 4 Backlog}

\begin{table}[!htbp]
  \centering
  \small
  \caption{Sprint 4 Backlog}
  \label{tab:backlog_sprint4}
  \begin{tabular}{|p{1cm}|p{5cm}|p{6cm}|p{2cm}|}
    \hline
    \textbf{ID} & \textbf{User Story} & \textbf{Description} & \textbf{Priority} \\ \hline
    US10 & View analytics & As an Admin, I want to view comprehensive analytics to understand transaction patterns and credit usage across my organization. & Critical \\ \hline
  \end{tabular}
\end{table}

\subsection{Use Case Diagram}

Figure~\ref{fig:uc_sprint4} presents the use case for Sprint 4: viewing analytics for comprehensive system insights.

\begin{figure}[H]
  \centering
  \includegraphics[width=0.8\textwidth]{images/usecase_sprint4.png}
  \caption{Use Case Diagram – Sprint 4}
  \label{fig:uc_sprint4}
\end{figure}

\subsection{Use Case Descriptions – Sprint 4}

In this section, we detail the use case identified for Sprint 4.

\subsubsection{Use Case "View analytics"}

\begin{longtable}{|p{0.2\textwidth}|p{0.75\textwidth}|}
  \caption{Description of use case "View analytics"}
  \label{tab:uc_view_analytics} \\
  \hline
  \textbf{Title} & View analytics \\ \hline
  \textbf{Actors} & Admin \\ \hline
  \textbf{Description} & Admin views comprehensive analytics to understand transaction patterns, credit usage, and organizational financial behavior. \\ \hline
  \textbf{Preconditions} & 
    \begin{itemize}[nosep,leftmargin=*]
      \item Admin is authenticated with analytics access permissions.
      \item Historical transaction data is available in the analytics system.
      \item Analytics controller is operational and responsive.
    \end{itemize} \\ \hline
  \textbf{Postconditions} & 
    \begin{itemize}[nosep,leftmargin=*]
      \item Analytics data is retrieved and displayed in interactive charts.
      \item Admin can filter and analyze spending patterns.
      \item Analytics viewing activity is logged for audit purposes.
    \end{itemize} \\ \hline
  \textbf{Main Scenario} &
    \begin{enumerate}[nosep,leftmargin=*]
      \item Admin requests to display analytics dashboard.
      \item Analytics interface submits request to analytics controller.
      \item Controller processes request and retrieves data.
      \item System returns analytics data to the interface.
      \item Interface displays data in charts and graphs.
      \item Admin can interact with analytics and apply filters.
    \end{enumerate} \\ \hline
  \textbf{Alternative Scenarios} &
    \begin{enumerate}[nosep,leftmargin=*]
      \item If no analytics data exists, display "No data available".
      \item If date range filter returns no results, show "No data for selected period".
    \end{enumerate} \\ \hline
  \textbf{Exception Scenarios} &
    \begin{enumerate}[nosep,leftmargin=*]
      \item If analytics service returns exception, pass error message to interface.
      \item If data processing fails, display "Failed to load analytics data".
      \item If network timeout occurs, show "Connection timeout, please retry".
    \end{enumerate} \\ \hline
\end{longtable}

\subsection{Sequence Diagrams}

\subsubsection{Sequence Diagram - View Analytics}

This diagram (see figure \ref{fig:seq_view_analytics}) models the analytics viewing process with data retrieval and display:

\begin{itemize}[nosep,leftmargin=*]
  \item Admin requests to display analytics data.
  \item Analytics interface submits request to analytics controller.
  \item Controller processes request and retrieves analytics data.
  \item System returns data for display in interactive charts.
  \item Error handling provides appropriate feedback for failures.
\end{itemize}

\begin{figure}[H] 
  \centering
  \includegraphics[width=0.9\textwidth,keepaspectratio]{images/seq_view_analytics.png}
  \caption{Sequence Diagram "View Analytics"}
  \label{fig:seq_view_analytics}
\end{figure}

\subsection{CI/CD Pipeline Implementation}

To ensure rapid and reliable delivery of Credix enhancements, we implemented comprehensive CI/CD pipelines for both frontend and backend components. The frontend uses Jenkins with GitHub webhooks, while the backend leverages Render platform for automated deployment.

\subsubsection{Frontend Pipeline (Next.js)}

The Jenkinsfile (see figure \ref{fig:jenkinsfile_credix}) located at the repository root defines the pipeline stages: code checkout, dependency installation, build, testing, and artifact generation.

\begin{figure}[H]
  \centering
  \includegraphics[width=0.9\textwidth]{images/jenkinsfile}
  \caption{Jenkinsfile in the Credix GitHub repository}
  \label{fig:jenkinsfile_credix}
\end{figure}

\textbf{GitHub Repository:}

On GitHub (see figure \ref{fig:github_webhook_credix}), the webhook configured on the main branch automatically triggers Jenkins on every push. The view below highlights the presence of the Jenkinsfile in the project root directory.

\begin{figure}[H]
  \centering
  \includegraphics[width=0.9\textwidth]{images/github_commit_trigger}
  \caption{GitHub repository view with Jenkinsfile and webhook configuration}
  \label{fig:github_webhook_credix}
\end{figure}

\textbf{Jenkins Dashboard:}

The Jenkins dashboard (see figure \ref{fig:jenkins_dashboard_credix}) provides a detailed view of each build's status, displaying the different pipeline stages and their status (green for success, red for failure).

\begin{figure}[H]
  \centering
  \includegraphics[width=0.9\textwidth]{images/jenkins_dashboard}
  \caption{Jenkins dashboard showing successful pipeline execution}
  \label{fig:jenkins_dashboard_credix}
\end{figure}

\textbf{Google Drive (Artifacts):}

Once the pipeline executes successfully, Jenkins automatically deposits the artifacts (build files for testing and deployment) into a shared folder on Google Drive (see figure \ref{fig:google_drive_artifacts}), facilitating their distribution within the team.

\begin{figure}[H]
  \centering
  \includegraphics[width=0.7\textwidth]{images/google_drive_artifacts}
  \caption{Google Drive folder containing generated build artifacts}
  \label{fig:google_drive_artifacts}
\end{figure}

\subsubsection{Backend Pipeline (Supabase Functions)}

Each push to the backend GitHub repository triggers our CI/CD pipeline on Render. The first step consists of building the Docker image of the service from the Dockerfile (see figure \ref{fig:dockerfile_credix}) included in the project.

\begin{figure}[H]
  \centering
  \includegraphics[width=0.8\textwidth]{images/dockerfile}
  \caption{Dockerfile excerpt used for building the backend image}
  \label{fig:dockerfile_credix}
\end{figure}

Once the commit is validated on GitHub (see figure \ref{fig:github_commit_trigger}), Render automatically detects the new revision and retrieves the code. On the GitHub interface (below), we can see the main branch and the triggering commit highlighted.

\begin{figure}[H]
  \centering
  \includegraphics[width=0.9\textwidth]{images/github_commit_trigger}
  \caption{GitHub repository view with CI/CD triggering commit}
  \label{fig:github_commit_trigger}
\end{figure}

Finally, Render (see figure \ref{fig:render_dashboard_credix}) builds the Docker image, deploys it on its infrastructure, and updates the production container. The Render dashboard displays the build history and their status (success or failure) in real-time.

\begin{figure}[H]
  \centering
  \includegraphics[width=0.9\textwidth]{images/render_dashboard}
  \caption{Render dashboard showing automatic backend deployment}
  \label{fig:render_dashboard_credix}
\end{figure}

Thanks to this organization, each contributor focuses on the functional quality of their code; Jenkins and Render handle the construction, testing, and production deployment. The result is a short, reliable, and fully traceable delivery cycle, meeting Credix's requirements for responsiveness and robustness.

\subsection{Realization (Sprint 4)}

This section presents the interfaces and features implemented during Sprint 4, focusing on analytics visualization and deployment capabilities. Each interface is illustrated with screenshots and accompanied by functionality descriptions.

\subsubsection{Analytics Dashboard}

The analytics dashboard provides comprehensive visualization of organizational transaction patterns and credit usage with interactive charts.

\begin{figure}[H]
  \centering
  \includegraphics[width=0.9\textwidth]{images/web_dashboard}
  \caption{Analytics Dashboard - Comprehensive Transaction Analytics}
  \label{fig:sprint4_analytics_dashboard}
\end{figure}

— \textbf{States presented:}
\begin{itemize}[nosep,leftmargin=*,label=•]
  \item \textbf{Interactive charts}  
    Visual representation of transaction patterns with drill-down capabilities (Linear, Bar, Pie, Doughnut charts).
  \item \textbf{Time period filters}  
    Selection of daily, weekly, monthly, or custom date ranges.
  \item \textbf{Category breakdown}  
    Analysis by transaction type and user groups.
  \item \textbf{Trend analysis}  
    Historical trends and usage insights.
\end{itemize}

— \textbf{Analytics viewing flow:}
\begin{itemize}[nosep,leftmargin=*,label=•]
  \item Admin accesses analytics dashboard.
  \item System retrieves and processes historical transaction data.
  \item Analytics controller compiles metrics and patterns.
  \item Interface displays interactive charts and filtering options.
  \item Admin can explore data through various visualization options.
\end{itemize}
