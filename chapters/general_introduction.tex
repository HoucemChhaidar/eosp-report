\chapter*{General Introduction}
\addcontentsline{toc}{chapter}{General Introduction}
\markboth{General Introduction}{General Introduction}
\adjustmtc

In recent years, the rapid advancement of information and communication technologies has brought about a profound transformation across nearly every sector of society. Among the most impacted domains is organizational resource management, where digitalization has reshaped how credits and financial resources are distributed, tracked, and managed within enterprises. As organizations increasingly seek efficiency, transparency, and control over internal credit allocation, web-based management systems have emerged as essential tools for streamlining these processes.

This shift is particularly evident in Tunisia, where digital transformation is accelerating across various organizational contexts. Yet, despite the promise of digital solutions, several technical and operational challenges persist—ranging from secure hierarchical credit distribution and transaction tracking to role-based access control and real-time system responsiveness.

In response to these evolving needs, this end-of-studies project proposes the design and implementation of a web-based credit management platform called Credix. The project aims to deliver a fully integrated system that connects an Angular-based web application with a Spring Boot backend and PostgreSQL database, enabling hierarchical credit distribution, wallet management, and transaction processing through a three-tier role system (SUPER\_ADMIN, ADMIN, and END\_USER).

Beyond its functional scope, the system addresses key challenges such as role-based access control, hierarchical credit distribution with configurable amounts, wallet-based transactions using barcodes, and comprehensive transaction history tracking. The solution supports enterprise-level features including credit distribution from Super Admins to Admins (after cash payment at ASM) and from Admins to end users, making it suitable for organizational deployment where controlled credit allocation is essential.

This thesis is organized into several chapters to provide a structured presentation of the project. The first chapter introduces the host organization and outlines the project objectives, followed by an analytical study of existing solutions. The second chapter elaborates on the proposed system architecture, detailing both functional and non-functional requirements, along with the rationale behind the technological choices made. The third and fourth chapters document the development process, divided into iterative sprints, highlighting implementation strategies, challenges encountered, and solutions applied.

The report concludes with a critical evaluation of the completed work, identifying potential areas for improvement and outlining perspectives for future enhancements.
